\chapter{III}

No-one dared to remove the apple lodged in Gregor's flesh, so it remained there as a visible reminder of his injury. He had suffered it there for more than a month, and his condition seemed serious enough to remind even his father that Gregor, despite his current sad and revolting form, was a family member who could not be treated as an enemy. On the contrary, as a family there was a duty to swallow any revulsion for him and to be patient, just to be patient.

Because of his injuries, Gregor had lost much of his mobility - probably permanently. He had been reduced to the condition of an ancient invalid and it took him long, long minutes to crawl across his room - crawling over the ceiling was out of the question - but this deterioration in his condition was fully (in his opinion) made up for by the door to the living room being left open every evening. He got into the habit of closely watching it for one or two hours before it was opened and then, lying in the darkness of his room where he could not be seen from the living room, he could watch the family in the light of the dinner table and listen to their conversation - with everyone's permission, in a way, and thus quite differently from before.

They no longer held the lively conversations of earlier times, of course, the ones that Gregor always thought about with longing when he was tired and getting into the damp bed in some small hotel room. All of them were usually very quiet nowadays. Soon after dinner, his father would go to sleep in his chair; his mother and sister would urge each other to be quiet; his mother, bent deeply under the lamp, would sew fancy underwear for a fashion shop; his sister, who had taken a sales job, learned shorthand and French in the evenings so that she might be able to get a better position later on. Sometimes his father would wake up and say to Gregor's mother "you're doing so much sewing again today!", as if he did not know that he had been dozing - and then he would go back to sleep again while mother and sister would exchange a tired grin.

With a kind of stubbornness, Gregor's father refused to take his uniform off even at home; while his nightgown hung unused on its peg Gregor's father would slumber where he was, fully dressed, as if always ready to serve and expecting to hear the voice of his superior even here. The uniform had not been new to start with, but as a result of this it slowly became even shabbier despite the efforts of Gregor's mother and sister to look after it. Gregor would often spend the whole evening looking at all the stains on this coat, with its gold buttons always kept polished and shiny, while the old man in it would sleep, highly uncomfortable but peaceful.

As soon as it struck ten, Gregor's mother would speak gently to his father to wake him and try to persuade him to go to bed, as he couldn't sleep properly where he was and he really had to get his sleep if he was to be up at six to get to work. But since he had been in work he had become more obstinate and would always insist on staying longer at the table, even though he regularly fell asleep and it was then harder than ever to persuade him to exchange the chair for his bed. Then, however much mother and sister would importune him with little reproaches and warnings he would keep slowly shaking his head for a quarter of an hour with his eyes closed and refusing to get up. Gregor's mother would tug at his sleeve, whisper endearments into his ear, Gregor's sister would leave her work to help her mother, but nothing would have any effect on him. He would just sink deeper into his chair. Only when the two women took him under the arms he would abruptly open his eyes, look at them one after the other and say: "What a life! This is what peace I get in my old age!" And supported by the two women he would lift himself up carefully as if he were carrying the greatest load himself, let the women take him to the door, send them off and carry on by himself while Gregor's mother would throw down her needle and his sister her pen so that they could run after his father and continue being of help to him.

Who, in this tired and overworked family, would have had time to give more attention to Gregor than was absolutely necessary? The household budget became even smaller; so now the maid was dismissed; an enormous, thick-boned charwoman with white hair that flapped around her head came every morning and evening to do the heaviest work; everything else was looked after by Gregor's mother on top of the large amount of sewing work she did. Gregor even learned, listening to the evening conversation about what price they had hoped for, that several items of jewellery belonging to the family had been sold, even though both mother and sister had been very fond of wearing them at functions and celebrations. But the loudest complaint was that although the flat was much too big for their present circumstances, they could not move out of it, there was no imaginable way of transferring Gregor to the new address. He could see quite well, though, that there were more reasons than consideration for him that made it difficult for them to move, it would have been quite easy to transport him in any suitable crate with a few air holes in it; the main thing holding the family back from their decision to move was much more to do with their total despair, and the thought that they had been struck with a misfortune unlike anything experienced by anyone else they knew or were related to. They carried out absolutely everything that the world expects from poor people, Gregor's father brought bank employees their breakfast, his mother sacrificed herself by washing clothes for strangers, his sister ran back and forth behind her desk at the behest of the customers, but they just did not have the strength to do any more. And the injury in Gregor's back began to hurt as much as when it was new. After they had come back from taking his father to bed Gregor's mother and sister would now leave their work where it was and sit close together, cheek to cheek; his mother would point to Gregor's room and say "Close that door, Grete", and then, when he was in the dark again, they would sit in the next room and their tears would mingle, or they would simply sit there staring dry-eyed at the table.

Gregor hardly slept at all, either night or day. Sometimes he would think of taking over the family's affairs, just like before, the next time the door was opened; he had long forgotten about his boss and the chief clerk, but they would appear again in his thoughts, the salesmen and the apprentices, that stupid teaboy, two or three friends from other businesses, one of the chambermaids from a provincial hotel, a tender memory that appeared and disappeared again, a cashier from a hat shop for whom his attention had been serious but too slow, - all of them appeared to him, mixed together with strangers and others he had forgotten, but instead of helping him and his family they were all of them inaccessible, and he was glad when they disappeared. Other times he was not at all in the mood to look after his family, he was filled with simple rage about the lack of attention he was shown, and although he could think of nothing he would have wanted, he made plans of how he could get into the pantry where he could take all the things he was entitled to, even if he was not hungry. Gregor's sister no longer thought about how she could please him but would hurriedly push some food or other into his room with her foot before she rushed out to work in the morning and at midday, and in the evening she would sweep it away again with the broom, indifferent as to whether it had been eaten or - more often than not - had been left totally untouched. She still cleared up the room in the evening, but now she could not have been any quicker about it. Smears of dirt were left on the walls, here and there were little balls of dust and filth. At first, Gregor went into one of the worst of these places when his sister arrived as a reproach to her, but he could have stayed there for weeks without his sister doing anything about it; she could see the dirt as well as he could but she had simply decided to leave him to it. At the same time she became touchy in a way that was quite new for her and which everyone in the family understood - cleaning up Gregor's room was for her and her alone. Gregor's mother did once thoroughly clean his room, and needed to use several bucketfuls of water to do it - although that much dampness also made Gregor ill and he lay flat on the couch, bitter and immobile. But his mother was to be punished still more for what she had done, as hardly had his sister arrived home in the evening than she noticed the change in Gregor's room and, highly aggrieved, ran back into the living room where, despite her mothers raised and imploring hands, she broke into convulsive tears. Her father, of course, was startled out of his chair and the two parents looked on astonished and helpless; then they, too, became agitated; Gregor's father, standing to the right of his mother, accused her of not leaving the cleaning of Gregor's room to his sister; from her left, Gregor's sister screamed at her that she was never to clean Gregor's room again; while his mother tried to draw his father, who was beside himself with anger, into the bedroom; his sister, quaking with tears, thumped on the table with her small fists; and Gregor hissed in anger that no-one had even thought of closing the door to save him the sight of this and all its noise.

Gregor's sister was exhausted from going out to work, and looking after Gregor as she had done before was even more work for her, but even so his mother ought certainly not to have taken her place. Gregor, on the other hand, ought not to be neglected. Now, though, the charwoman was here. This elderly widow, with a robust bone structure that made her able to withstand the hardest of things in her long life, wasn't really repelled by Gregor. Just by chance one day, rather than any real curiosity, she opened the door to Gregor's room and found herself face to face with him. He was taken totally by surprise, no-one was chasing him but he began to rush to and fro while she just stood there in amazement with her hands crossed in front of her. From then on she never failed to open the door slightly every evening and morning and look briefly in on him. At first she would call to him as she did so with words that she probably considered friendly, such as "come on then, you old dung-beetle!", or "look at the old dung-beetle there!" Gregor never responded to being spoken to in that way, but just remained where he was without moving as if the door had never even been opened. If only they had told this charwoman to clean up his room every day instead of letting her disturb him for no reason whenever she felt like it! One day, early in the morning while a heavy rain struck the windowpanes, perhaps indicating that spring was coming, she began to speak to him in that way once again. Gregor was so resentful of it that he started to move toward her, he was slow and infirm, but it was like a kind of attack. Instead of being afraid, the charwoman just lifted up one of the chairs from near the door and stood there with her mouth open, clearly intending not to close her mouth until the chair in her hand had been slammed down into Gregor's back. "Aren't you coming any closer, then?", she asked when Gregor turned round again, and she calmly put the chair back in the corner.

Gregor had almost entirely stopped eating. Only if he happened to find himself next to the food that had been prepared for him he might take some of it into his mouth to play with it, leave it there a few hours and then, more often than not, spit it out again. At first he thought it was distress at the state of his room that stopped him eating, but he had soon got used to the changes made there. They had got into the habit of putting things into this room that they had no room for anywhere else, and there were now many such things as one of the rooms in the flat had been rented out to three gentlemen. These earnest gentlemen - all three of them had full beards, as Gregor learned peering through the crack in the door one day - were painfully insistent on things' being tidy. This meant not only in their own room but, since they had taken a room in this establishment, in the entire flat and especially in the kitchen. Unnecessary clutter was something they could not tolerate, especially if it was dirty. They had moreover brought most of their own furnishings and equipment with them. For this reason, many things had become superfluous which, although they could not be sold, the family did not wish to discard. All these things found their way into Gregor's room. The dustbins from the kitchen found their way in there too. The charwoman was always in a hurry, and anything she couldn't use for the time being she would just chuck in there. He, fortunately, would usually see no more than the object and the hand that held it. The woman most likely meant to fetch the things back out again when she had time and the opportunity, or to throw everything out in one go, but what actually happened was that they were left where they landed when they had first been thrown unless Gregor made his way through the junk and moved it somewhere else. At first he moved it because, with no other room free where he could crawl about, he was forced to, but later on he came to enjoy it although moving about in that way left him sad and tired to death, and he would remain immobile for hours afterwards.

The gentlemen who rented the room would sometimes take their evening meal at home in the living room that was used by everyone, and so the door to this room was often kept closed in the evening. But Gregor found it easy to give up having the door open, he had, after all, often failed to make use of it when it was open and, without the family having noticed it, lain in his room in its darkest corner. One time, though, the charwoman left the door to the living room slightly open, and it remained open when the gentlemen who rented the room came in in the evening and the light was put on. They sat up at the table where, formerly, Gregor had taken his meals with his father and mother, they unfolded the serviettes and picked up their knives and forks. Gregor's mother immediately appeared in the doorway with a dish of meat and soon behind her came his sister with a dish piled high with potatoes. The food was steaming, and filled the room with its smell. The gentlemen bent over the dishes set in front of them as if they wanted to test the food before eating it, and the gentleman in the middle, who seemed to count as an authority for the other two, did indeed cut off a piece of meat while it was still in its dish, clearly wishing to establish whether it was sufficiently cooked or whether it should be sent back to the kitchen. It was to his satisfaction, and Gregor's mother and sister, who had been looking on anxiously, began to breathe again and smiled.

The family themselves ate in the kitchen. Nonetheless, Gregor's father came into the living room before he went into the kitchen, bowed once with his cap in his hand and did his round of the table. The gentlemen stood as one, and mumbled something into their beards. Then, once they were alone, they ate in near perfect silence. It seemed remarkable to Gregor that above all the various noises of eating their chewing teeth could still be heard, as if they had wanted to show Gregor that you need teeth in order to eat and it was not possible to perform anything with jaws that are toothless however nice they might be. "I'd like to eat something", said Gregor anxiously, "but not anything like they're eating. They do feed themselves. And here I am, dying!"

Throughout all this time, Gregor could not remember having heard the violin being played, but this evening it began to be heard from the kitchen. The three gentlemen had already finished their meal, the one in the middle had produced a newspaper, given a page to each of the others, and now they leant back in their chairs reading them and smoking. When the violin began playing they became attentive, stood up and went on tip-toe over to the door of the hallway where they stood pressed against each other. Someone must have heard them in the kitchen, as Gregor's father called out: "Is the playing perhaps unpleasant for the gentlemen? We can stop it straight away." "On the contrary", said the middle gentleman, "would the young lady not like to come in and play for us here in the room, where it is, after all, much more cosy and comfortable?" "Oh yes, we'd love to", called back Gregor's father as if he had been the violin player himself. The gentlemen stepped back into the room and waited. Gregor's father soon appeared with the music stand, his mother with the music and his sister with the violin. She calmly prepared everything for her to begin playing; his parents, who had never rented a room out before and therefore showed an exaggerated courtesy towards the three gentlemen, did not even dare to sit on their own chairs; his father leant against the door with his right hand pushed in between two buttons on his uniform coat; his mother, though, was offered a seat by one of the gentlemen and sat - leaving the chair where the gentleman happened to have placed it - out of the way in a corner.

His sister began to play; father and mother paid close attention, one on each side, to the movements of her hands. Drawn in by the playing, Gregor had dared to come forward a little and already had his head in the living room. Before, he had taken great pride in how considerate he was but now it hardly occurred to him that he had become so thoughtless about the others. What's more, there was now all the more reason to keep himself hidden as he was covered in the dust that lay everywhere in his room and flew up at the slightest movement; he carried threads, hairs, and remains of food about on his back and sides; he was much too indifferent to everything now to lay on his back and wipe himself on the carpet like he had used to do several times a day. And despite this condition, he was not too shy to move forward a little onto the immaculate floor of the living room.

No-one noticed him, though. The family was totally preoccupied with the violin playing; at first, the three gentlemen had put their hands in their pockets and come up far too close behind the music stand to look at all the notes being played, and they must have disturbed Gregor's sister, but soon, in contrast with the family, they withdrew back to the window with their heads sunk and talking to each other at half volume, and they stayed by the window while Gregor's father observed them anxiously. It really now seemed very obvious that they had expected to hear some beautiful or entertaining violin playing but had been disappointed, that they had had enough of the whole performance and it was only now out of politeness that they allowed their peace to be disturbed. It was especially unnerving, the way they all blew the smoke from their cigarettes upwards from their mouth and noses. Yet Gregor's sister was playing so beautifully. Her face was leant to one side, following the lines of music with a careful and melancholy expression. Gregor crawled a little further forward, keeping his head close to the ground so that he could meet her eyes if the chance came. Was he an animal if music could captivate him so? It seemed to him that he was being shown the way to the unknown nourishment he had been yearning for. He was determined to make his way forward to his sister and tug at her skirt to show her she might come into his room with her violin, as no-one appreciated her playing here as much as he would. He never wanted to let her out of his room, not while he lived, anyway; his shocking appearance should, for once, be of some use to him; he wanted to be at every door of his room at once to hiss and spit at the attackers; his sister should not be forced to stay with him, though, but stay of her own free will; she would sit beside him on the couch with her ear bent down to him while he told her how he had always intended to send her to the conservatory, how he would have told everyone about it last Christmas - had Christmas really come and gone already? - if this misfortune hadn't got in the way, and refuse to let anyone dissuade him from it. On hearing all this, his sister would break out in tears of emotion, and Gregor would climb up to her shoulder and kiss her neck, which, since she had been going out to work, she had kept free without any necklace or collar.

"Mr. Samsa!", shouted the middle gentleman to Gregor's father, pointing, without wasting any more words, with his forefinger at Gregor as he slowly moved forward. The violin went silent, the middle of the three gentlemen first smiled at his two friends, shaking his head, and then looked back at Gregor. His father seemed to think it more important to calm the three gentlemen before driving Gregor out, even though they were not at all upset and seemed to think Gregor was more entertaining than the violin playing had been. He rushed up to them with his arms spread out and attempted to drive them back into their room at the same time as trying to block their view of Gregor with his body. Now they did become a little annoyed, and it was not clear whether it was his father's behaviour that annoyed them or the dawning realisation that they had had a neighbour like Gregor in the next room without knowing it. They asked Gregor's father for explanations, raised their arms like he had, tugged excitedly at their beards and moved back towards their room only very slowly. Meanwhile Gregor's sister had overcome the despair she had fallen into when her playing was suddenly interrupted. She had let her hands drop and let violin and bow hang limply for a while but continued to look at the music as if still playing, but then she suddenly pulled herself together, lay the instrument on her mother's lap who still sat laboriously struggling for breath where she was, and ran into the next room which, under pressure from her father, the three gentlemen were more quickly moving toward. Under his sister's experienced hand, the pillows and covers on the beds flew up and were put into order and she had already finished making the beds and slipped out again before the three gentlemen had reached the room. Gregor's father seemed so obsessed with what he was doing that he forgot all the respect he owed to his tenants. He urged them and pressed them until, when he was already at the door of the room, the middle of the three gentlemen shouted like thunder and stamped his foot and thereby brought Gregor's father to a halt. "I declare here and now", he said, raising his hand and glancing at Gregor's mother and sister to gain their attention too, "that with regard to the repugnant conditions that prevail in this flat and with this family" - here he looked briefly but decisively at the floor - "I give immediate notice on my room. For the days that I have been living here I will, of course, pay nothing at all, on the contrary I will consider whether to proceed with some kind of action for damages from you, and believe me it would be very easy to set out the grounds for such an action." He was silent and looked straight ahead as if waiting for something. And indeed, his two friends joined in with the words: "And we also give immediate notice." With that, he took hold of the door handle and slammed the door.

Gregor's father staggered back to his seat, feeling his way with his hands, and fell into it; it looked as if he was stretching himself out for his usual evening nap but from the uncontrolled way his head kept nodding it could be seen that he was not sleeping at all. Throughout all this, Gregor had lain still where the three gentlemen had first seen him. His disappointment at the failure of his plan, and perhaps also because he was weak from hunger, made it impossible for him to move. He was sure that everyone would turn on him any moment, and he waited. He was not even startled out of this state when the violin on his mother's lap fell from her trembling fingers and landed loudly on the floor.

"Father, Mother", said his sister, hitting the table with her hand as introduction, "we can't carry on like this. Maybe you can't see it, but I can. I don't want to call this monster my brother, all I can say is: we have to try and get rid of it. We've done all that's humanly possible to look after it and be patient, I don't think anyone could accuse us of doing anything wrong."

"She's absolutely right", said Gregor's father to himself. His mother, who still had not had time to catch her breath, began to cough dully, her hand held out in front of her and a deranged expression in her eyes.

Gregor's sister rushed to his mother and put her hand on her forehead. Her words seemed to give Gregor's father some more definite ideas. He sat upright, played with his uniform cap between the plates left by the three gentlemen after their meal, and occasionally looked down at Gregor as he lay there immobile.

"We have to try and get rid of it", said Gregor's sister, now speaking only to her father, as her mother was too occupied with coughing to listen, "it'll be the death of both of you, I can see it coming. We can't all work as hard as we have to and then come home to be tortured like this, we can't endure it. I can't endure it any more." And she broke out so heavily in tears that they flowed down the face of her mother, and she wiped them away with mechanical hand movements.

"My child", said her father with sympathy and obvious understanding, "what are we to do?"

His sister just shrugged her shoulders as a sign of the helplessness and tears that had taken hold of her, displacing her earlier certainty.

"If he could just understand us", said his father almost as a question; his sister shook her hand vigorously through her tears as a sign that of that there was no question.

"If he could just understand us", repeated Gregor's father, closing his eyes in acceptance of his sister's certainty that that was quite impossible, "then perhaps we could come to some kind of arrangement with him. But as it is ..."

"It's got to go", shouted his sister, "that's the only way, Father. You've got to get rid of the idea that that's Gregor. We've only harmed ourselves by believing it for so long. How can that be Gregor? If it were Gregor he would have seen long ago that it's not possible for human beings to live with an animal like that and he would have gone of his own free will. We wouldn't have a brother any more, then, but we could carry on with our lives and remember him with respect. As it is this animal is persecuting us, it's driven out our tenants, it obviously wants to take over the whole flat and force us to sleep on the streets. Father, look, just look", she suddenly screamed, "he's starting again!"   In her alarm, which was totally beyond Gregor's comprehension, his sister even abandoned his mother as she pushed herself vigorously out of her chair as if more willing to sacrifice her own mother than stay anywhere near Gregor. She rushed over to behind her father, who had become excited merely because she was and stood up half raising his hands in front of Gregor's sister as if to protect her.

But Gregor had had no intention of frightening anyone, least of all his sister. All he had done was begin to turn round so that he could go back into his room, although that was in itself quite startling as his pain-wracked condition meant that turning round required a great deal of effort and he was using his head to help himself do it, repeatedly raising it and striking it against the floor. He stopped and looked round. They seemed to have realised his good intention and had only been alarmed briefly. Now they all looked at him in unhappy silence. His mother lay in her chair with her legs stretched out and pressed against each other, her eyes nearly closed with exhaustion; his sister sat next to his father with her arms around his neck.

"Maybe now they'll let me turn round", thought Gregor and went back to work. He could not help panting loudly with the effort and had sometimes to stop and take a rest. No-one was making him rush any more, everything was left up to him. As soon as he had finally finished turning round he began to move straight ahead. He was amazed at the great distance that separated him from his room, and could not understand how he had covered that distance in his weak state a little while before and almost without noticing it. He concentrated on crawling as fast as he could and hardly noticed that there was not a word, not any cry, from his family to distract him. He did not turn his head until he had reached the doorway. He did not turn it all the way round as he felt his neck becoming stiff, but it was nonetheless enough to see that nothing behind him had changed, only his sister had stood up. With his last glance he saw that his mother had now fallen completely asleep.

He was hardly inside his room before the door was hurriedly shut, bolted and locked. The sudden noise behind Gregor so startled him that his little legs collapsed under him. It was his sister who had been in so much of a rush. She had been standing there waiting and sprung forward lightly, Gregor had not heard her coming at all, and as she turned the key in the lock she said loudly to her parents "At last!".

"What now, then?", Gregor asked himself as he looked round in the darkness. He soon made the discovery that he could no longer move at all. This was no surprise to him, it seemed rather that being able to actually move around on those spindly little legs until then was unnatural. He also felt relatively comfortable. It is true that his entire body was aching, but the pain seemed to be slowly getting weaker and weaker and would finally disappear altogether. He could already hardly feel the decayed apple in his back or the inflamed area around it, which was entirely covered in white dust. He thought back of his family with emotion and love. If it was possible, he felt that he must go away even more strongly than his sister. He remained in this state of empty and peaceful rumination until he heard the clock tower strike three in the morning. He watched as it slowly began to get light everywhere outside the window too. Then, without his willing it, his head sank down completely, and his last breath flowed weakly from his nostrils.

When the cleaner came in early in the morning - they'd often asked her not to keep slamming the doors but with her strength and in her hurry she still did, so that everyone in the flat knew when she'd arrived and from then on it was impossible to sleep in peace - she made her usual brief look in on Gregor and at first found nothing special. She thought he was laying there so still on purpose, playing the martyr; she attributed all possible understanding to him. She happened to be holding the long broom in her hand, so she tried to tickle Gregor with it from the doorway. When she had no success with that she tried to make a nuisance of herself and poked at him a little, and only when she found she could shove him across the floor with no resistance at all did she start to pay attention. She soon realised what had really happened, opened her eyes wide, whistled to herself, but did not waste time to yank open the bedroom doors and shout loudly into the darkness of the bedrooms: "Come and 'ave a look at this, it's dead, just lying there, stone dead!"

Mr. and Mrs. Samsa sat upright there in their marriage bed and had to make an effort to get over the shock caused by the cleaner before they could grasp what she was saying. But then, each from his own side, they hurried out of bed. Mr. Samsa threw the blanket over his shoulders, Mrs. Samsa just came out in her nightdress; and that is how they went into Gregor's room. On the way they opened the door to the living room where Grete had been sleeping since the three gentlemen had moved in; she was fully dressed as if she had never been asleep, and the paleness of her face seemed to confirm this. "Dead?", asked Mrs. Samsa, looking at the charwoman enquiringly, even though she could have checked for herself and could have known it even without checking. "That's what I said", replied the cleaner, and to prove it she gave Gregor's body another shove with the broom, sending it sideways across the floor. Mrs. Samsa made a movement as if she wanted to hold back the broom, but did not complete it. "Now then", said Mr. Samsa, "let's give thanks to God for that". He crossed himself, and the three women followed his example. Grete, who had not taken her eyes from the corpse, said: "Just look how thin he was. He didn't eat anything for so long. The food came out again just the same as when it went in". Gregor's body was indeed completely dried up and flat, they had not seen it until then, but now he was not lifted up on his little legs, nor did he do anything to make them look away.

"Grete, come with us in here for a little while", said Mrs. Samsa with a pained smile, and Grete followed her parents into the bedroom but not without looking back at the body. The cleaner shut the door and opened the window wide. Although it was still early in the morning the fresh air had something of warmth mixed in with it. It was already the end of March, after all.

The three gentlemen stepped out of their room and looked round in amazement for their breakfasts; they had been forgotten about. "Where is our breakfast?", the middle gentleman asked the cleaner irritably. She just put her finger on her lips and made a quick and silent sign to the men that they might like to come into Gregor's room. They did so, and stood around Gregor's corpse with their hands in the pockets of their well-worn coats. It was now quite light in the room.

Then the door of the bedroom opened and Mr. Samsa appeared in his uniform with his wife on one arm and his daughter on the other. All of them had been crying a little; Grete now and then pressed her face against her father's arm.

"Leave my home. Now!", said Mr. Samsa, indicating the door and without letting the women from him. "What do you mean?", asked the middle of the three gentlemen somewhat disconcerted, and he smiled sweetly. The other two held their hands behind their backs and continually rubbed them together in gleeful anticipation of a loud quarrel which could only end in their favour. "I mean just what I said", answered Mr. Samsa, and, with his two companions, went in a straight line towards the man. At first, he stood there still, looking at the ground as if the contents of his head were rearranging themselves into new positions. "Alright, we'll go then", he said, and looked up at Mr. Samsa as if he had been suddenly overcome with humility and wanted permission again from Mr. Samsa for his decision. Mr. Samsa merely opened his eyes wide and briefly nodded to him several times. At that, and without delay, the man actually did take long strides into the front hallway; his two friends had stopped rubbing their hands some time before and had been listening to what was being said. Now they jumped off after their friend as if taken with a sudden fear that Mr. Samsa might go into the hallway in front of them and break the connection with their leader. Once there, all three took their hats from the stand, took their sticks from the holder, bowed without a word and left the premises. Mr. Samsa and the two women followed them out onto the landing; but they had had no reason to mistrust the men's intentions and as they leaned over the landing they saw how the three gentlemen made slow but steady progress down the many steps. As they turned the corner on each floor they disappeared and would reappear a few moments later; the further down they went, the more that the Samsa family lost interest in them; when a butcher's boy, proud of posture with his tray on his head, passed them on his way up and came nearer than they were, Mr. Samsa and the women came away from the landing and went, as if relieved, back into the flat.

They decided the best way to make use of that day was for relaxation and to go for a walk; not only had they earned a break from work but they were in serious need of it. So they sat at the table and wrote three letters of excusal, Mr. Samsa to his employers, Mrs. Samsa to her contractor and Grete to her principal. The cleaner came in while they were writing to tell them she was going, she'd finished her work for that morning. The three of them at first just nodded without looking up from what they were writing, and it was only when the cleaner still did not seem to want to leave that they looked up in irritation. "Well?", asked Mr. Samsa. The charwoman stood in the doorway with a smile on her face as if she had some tremendous good news to report, but would only do it if she was clearly asked to. The almost vertical little ostrich feather on her hat, which had been a source of irritation to Mr. Samsa all the time she had been working for them, swayed gently in all directions. "What is it you want then?", asked Mrs. Samsa, whom the cleaner had the most respect for. "Yes", she answered, and broke into a friendly laugh that made her unable to speak straight away, "well then, that thing in there, you needn't worry about how you're going to get rid of it. That's all been sorted out."   Mrs. Samsa and Grete bent down over their letters as if intent on continuing with what they were writing; Mr. Samsa saw that the cleaner wanted to start describing everything in detail but, with outstretched hand, he made it quite clear that she was not to. So, as she was prevented from telling them all about it, she suddenly remembered what a hurry she was in and, clearly peeved, called out "Cheerio then, everyone", turned round sharply and left, slamming the door terribly as she went.

"Tonight she gets sacked", said Mr. Samsa, but he received no reply from either his wife or his daughter as the charwoman seemed to have destroyed the peace they had only just gained. They got up and went over to the window where they remained with their arms around each other. Mr. Samsa twisted round in his chair to look at them and sat there watching for a while. Then he called out: "Come here, then. Let's forget about all that old stuff, shall we. Come and give me a bit of attention". The two women immediately did as he said, hurrying over to him where they kissed him and hugged him and then they quickly finished their letters.

After that, the three of them left the flat together, which was something they had not done for months, and took the tram out to the open country outside the town. They had the tram, filled with warm sunshine, all to themselves. Leant back comfortably on their seats, they discussed their prospects and found that on closer examination they were not at all bad - until then they had never asked each other about their work but all three had jobs which were very good and held particularly good promise for the future. The greatest improvement for the time being, of course, would be achieved quite easily by moving house; what they needed now was a flat that was smaller and cheaper than the current one which had been chosen by Gregor, one that was in a better location and, most of all, more practical. All the time, Grete was becoming livelier. With all the worry they had been having of late her cheeks had become pale, but, while they were talking, Mr. and Mrs. Samsa were struck, almost simultaneously, with the thought of how their daughter was blossoming into a well built and beautiful young lady. They became quieter. Just from each other's glance and almost without knowing it they agreed that it would soon be time to find a good man for her. And, as if in confirmation of their new dreams and good intentions, as soon as they reached their destination Grete was the first to get up and stretch out her young body.